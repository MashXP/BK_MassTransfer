
% --- Custom Commands for Variables ---
\newcommand{\Linert}{L'}   % Molar Liquid Rate (Inert)
\newcommand{\Vinert}{V'}   % Molar Gas Rate (Inert)
\newcommand{\pA}{p_A}      % Partial Pressure of Solute A
\newcommand{\Hconst}{H}    % Henry's Law Constant
\newcommand{\mEq}{m}       % Equilibrium constant (y=mx)
\newcommand{\AbsFactor}{A} % Absorption Factor
\newcommand{\Nstages}{N}   % Number of theoretical stages

\section{Absorption - \\ Fundamentals of Gas Absorption}

\subsection{Core Concepts}
\begin{itemize}
    \item \textbf{Absorption:} A mass transfer operation where one or more soluble components (solutes) are removed from a gas phase by dissolving them into a liquid phase (solvent or absorbent).
    \item \textbf{Desorption or Stripping:} The reverse of absorption, where a solute is transferred from the liquid phase to the gas phase. It is often used to regenerate the solvent.
    \item \textbf{Key Components:}
    \begin{itemize}
        \item \textbf{Solute (A):} The component being transferred from the gas to the liquid.
        \item \textbf{Inert Gas (B):} The bulk gas component, assumed to be insoluble in the liquid.
        \item \textbf{Solvent (C):} The liquid used to absorb the solute, assumed to be non-volatile.
    \end{itemize}
\end{itemize}

\section{Gas-Liquid Equilibrium (GLE)}
The relationship between the solute concentration in the gas phase and the liquid phase at equilibrium is crucial for design. For dilute systems, this is described by Henry's Law.

\subsection{Henry's Law}
\begin{itemize}
    \item \textbf{Definition:} At a constant temperature, the amount of a gas that dissolves in a liquid is directly proportional to the partial pressure of that gas in equilibrium with the liquid.
    \item \textbf{Pressure Form:}
    \begin{equation}
        \pA = \Hconst \cdot \xVar_A
    \end{equation}
    where:
    \begin{itemize}
        \item $\pA$ is the partial pressure of solute A in the gas phase (e.g., atm, kPa).
        \item $\Hconst$ is the Henry's Law constant (e.g., atm/mole fraction).
        \item $\xVar_A$ is the mole fraction of solute A in the liquid phase.
    \end{itemize}
    \item \textbf{Mole Fraction Form:} The equilibrium relationship is often expressed in terms of mole fractions. Since $\yVar_A = \pA / P_{\text{total}}$:
    \begin{equation}
        \yVar_A = \mEq \cdot \xVar_A
    \end{equation}
    where:
    \begin{itemize}
        \item $\yVar_A$ is the mole fraction of solute A in the gas phase at equilibrium.
        \item $\mEq$ is the equilibrium distribution coefficient, equal to $\Hconst/P_{\text{total}}$. This is the slope of the equilibrium line on an x-y diagram.
    \end{itemize}
\end{itemize}

\section{Single-Stage Equilibrium Contact}
A single unit where gas and liquid streams are intimately mixed, reach equilibrium, and are then separated.

\subsection{Material Balances}
Using inert (solute-free) molar flow rates is convenient.

\begin{itemize}
    \item \textbf{Definitions:}
    \begin{itemize}
        \item $\Linert$: Molar flow rate of inert solvent C (kg-mol/h).
        \item $\Vinert$: Molar flow rate of inert gas B (kg-mol/h).
        \item The mole ratios are defined as:
        \[
            X_A = \frac{\xVar_A}{1 - \xVar_A}, \qquad 
            Y_A = \frac{\yVar_A}{1 - \yVar_A}
        \]
    \end{itemize}

    \item \textbf{Component A Balance (Mole Ratio Basis):} This forms the operating line for a single stage.
    \begin{equation}
        \Linert X_{A,\text{in}} + \Vinert Y_{A,\text{in}} = \Linert X_{A,\text{out}} + \Vinert Y_{A,\text{out}}
    \end{equation}
    where the outlet streams are in equilibrium: $\yVar_{A,\text{out}} = \mEq \xVar_{A,\text{out}}$.
\end{itemize}

\section{Multi-Stage Countercurrent Absorption}
A series of stages are arranged so that the gas and liquid flow in opposite directions.

\subsection{Material Balances and the Operating Line}

\begin{itemize}
    \item \textbf{Tower Notation:}
    \begin{itemize}
        \item $\Lvar_{\text{in}}, \xVar_{A,\text{in}}$: Inlet liquid (solvent) stream entering the top.
        \item $\Vvar_{\text{in}}, \yVar_{A,\text{in}}$: Inlet gas stream entering the bottom.
        \item $\Lvar_{\text{out}}, \xVar_{A,\text{out}}$: Outlet liquid stream leaving the bottom.
        \item $\Vvar_{\text{out}}, \yVar_{A,\text{out}}$: Outlet gas stream leaving the top.
    \end{itemize}

    \item \textbf{Overall Component Balance (Mole Ratio Basis):}
    \begin{equation}
        \Linert X_{A,\text{in}} + \Vinert Y_{A,\text{in}} = \Linert X_{A,\text{out}} + \Vinert Y_{A,\text{out}}
    \end{equation}

    \item \textbf{Operating Line for Dilute Systems:} If total molar flow rates are approximately constant, the operating line is a straight line on the $x$–$y$ diagram that relates the compositions of passing streams at any point in the column. It connects the conditions at the top and bottom of the tower.
\end{itemize}

\subsection{Graphical Method for Number of Stages}
A graphical procedure on the $x$–$y$ diagram determines the number of theoretical stages.

\begin{enumerate}
    \item \textbf{Plot Data:} Plot the equilibrium line ($\yVar = \mEq \xVar$) and the operating line. The operating line connects the tower's top conditions $(\xVar_{A,\text{in}}, \yVar_{A,\text{out}})$ with its bottom conditions $(\xVar_{A,\text{out}}, \yVar_{A,\text{in}})$.
    \item \textbf{Step Off Stages:} Starting from the bottom point $(\xVar_{A,\text{out}}, \yVar_{A,\text{in}})$ on the operating line, draw a horizontal line to the equilibrium curve, then a vertical line back to the operating line. This completes one theoretical stage.
    \item \textbf{Repeat:} Continue this "staircase" construction until the liquid mole fraction is less than or equal to the desired inlet composition $\xVar_{A,\text{in}}$.
    \item \textbf{Count Stages:} The number of steps gives the number of theoretical stages, $\Nstages$.
\end{enumerate}

\subsection{Analytical Method (Kremser Equation)}
For cases where both the operating and equilibrium lines are straight.

\begin{itemize}
    \item \textbf{Absorption Factor ($\AbsFactor$):} A dimensionless parameter indicating absorption effectiveness.
    \begin{equation}
        \AbsFactor = \frac{\Lvar}{\mEq \Vvar} 
        = \frac{\text{Slope of Operating Line}}{\text{Slope of Equilibrium Line}}
    \end{equation}
    For efficient absorption, $\AbsFactor > 1$, typically between 1.2 and 2.0.

    \item \textbf{Kremser Equation for Absorption:}
    \begin{equation}
        \Nstages 
        = 
        \frac{
            \ln \left[
                \left( 
                    \frac{\yVar_{A,\text{in}} - \mEq \xVar_{A,\text{in}}}
                         {\yVar_{A,\text{out}} - \mEq \xVar_{A,\text{in}}}
                \right)
                \left( 1 - \frac{1}{\AbsFactor} \right)
                + \frac{1}{\AbsFactor}
            \right]
        }{
            \ln(\AbsFactor)
        }
    \end{equation}
    This equation is valid for $\AbsFactor \neq 1$.

    \item \textbf{Non-Constant Flow Rates:} A geometric average absorption factor may be used:
    \begin{equation}
    \begin{gathered}
        \AbsFactor_{\text{avg}} = \sqrt{\AbsFactor_{\text{top}} \, \AbsFactor_{\text{bottom}}} \\
            \\ \quad \AbsFactor_{\text{top}} = \frac{\Lvar_{\text{in}}}{\mEq_{\text{top}} \Vvar_{\text{out}}} 
            \\ \quad \AbsFactor_{\text{bottom}} = \frac{\Lvar_{\text{out}}}{\mEq_{\text{bottom}} \Vvar_{\text{in}}}
    \end{gathered}
    \end{equation}
\end{itemize}

\section{Tray and Column Efficiency}
Theoretical stages are an ideal concept. The performance of a real column requires accounting for efficiency.
\begin{itemize}
    \item \textbf{Overall Column Efficiency ($E_o$):} Relates theoretical stages to actual trays.
    \begin{equation}
        \text{Number of Actual Trays} = \frac{\text{Number of Theoretical Stages}}{E_o}
    \end{equation}
\end{itemize}