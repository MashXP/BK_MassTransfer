
% Custom Commands based on Crystallization Context
\newcommand{\Conc}{C}                     % Concentration
\newcommand{\SatConc}{C^*}                % Saturation Concentration
\newcommand{\Temp}{T}                     % Temperature
\newcommand{\FeedMass}{L}                 % Mass of Feed Solution
\newcommand{\SolMass}{S}                  % Mass of Saturated Solution (Mother Liquor)
\newcommand{\CrysMass}{C_{rys}}           % Mass of Crystals
\newcommand{\VaporMass}{W}                % Mass of Vapor (Evaporated Water)
\newcommand{\MassFrac}[2]{x_{#1,#2}}      % Mass fraction of component #1 in stream #2
\newcommand{\Enthalpy}{H}                 % Enthalpy
\newcommand{\HeatTrans}{q}                % Total Heat Transferred
\newcommand{\HeatSol}{\Delta H_{sol}}     % Heat of Solution
\newcommand{\HeatCrys}{\Delta H_{crys}}   % Heat of Crystallization

\section{Introduction to Crystallization}
Crystallization is a specific mass transfer process involving the formation of solid crystals from a homogeneous solution. It is widely used to separate and purify substances. The two-phase mixture of mother liquor and crystals in the crystallizer is often called **magma**.

\section{Solubility and Crystal Formation}
The process relies on the relationship between solute concentration and temperature, often visualized using a solubility curve.

\subsection{Regions of the Solubility Curve}
\begin{itemize}
    \item \textbf{Undersaturated Region (Point A):} The solute concentration is lower than the equilibrium saturation concentration ($\Conc < \SatConc(\Temp)$). Crystals will dissolve rather than form in this region.
    \item \textbf{Supersaturated Region (Point B):} The concentration exceeds the equilibrium solubility ($\Conc > \SatConc(\Temp)$). This state is achieved by cooling a saturated solution or evaporating solvent. Supersaturation is the driving force for crystallization.
    \item \textbf{Metastable Zone:} A specific region of supersaturation where spontaneous nucleation does not occur immediately. Crystallization in this zone often requires "seeding" (adding small crystal particles) to initiate the process.
\end{itemize}

\subsection{Mechanisms of Formation}
The lifecycle of a crystal generally follows these steps:
\begin{enumerate}
    \item \textbf{Nucleation:} The birth of new solid particles (nuclei) from the liquid phase. This usually occurs at the metastable limit. Factors influencing the metastable limit include agitation, impurities, and cooling rate. Atmospheric pressure typically has a negligible effect.
    \item \textbf{Crystal Growth:} Solute molecules diffuse from the bulk solution to the crystal surface and integrate into the crystal lattice, reducing the supersaturation.
\end{enumerate}

\section{Yield and Material Balances}
To determine the quantity of crystals produced (yield), a mass balance is performed around the crystallizer unit.

\subsection{General Mass Balance}
The law of conservation of mass is applied to the entire system. For a steady-state process:
\begin{equation}
    \text{Input} = \text{Output}
\end{equation}
\begin{equation}
    \FeedMass \cdot \MassFrac{i}{L} = \SolMass \cdot \MassFrac{i}{S} + \VaporMass \cdot \MassFrac{i}{W} + \CrysMass \cdot \MassFrac{i}{C}
\end{equation}
Where:
\begin{itemize}
    \item $\FeedMass$: Mass of the feed solution.
    \item $\SolMass$: Mass of the mother liquor (saturated solution leaving the unit).
    \item $\VaporMass$: Mass of vapor/water evaporated (often zero in simple cooling crystallization).
    \item $\CrysMass$: Mass of wet crystals produced.
    \item $\MassFrac{i}{stream}$: Mass fraction of component $i$ (solute or solvent) in the respective stream.
\end{itemize}
To solve for unknowns (typically $\SolMass$ and $\CrysMass$), two independent equations are required (e.g., a Solute Balance and a Solvent/Water Balance).

\subsection{Calculations Involving Hydrates}
Many salts crystallize as hydrates (e.g., $Na_2CO_3 \cdot 10H_2O$). The water of crystallization must be accounted for in the solute balance.
\begin{itemize}
    \item \textbf{Solute Fraction in Crystal ($\MassFrac{solute}{C}$):}
    \begin{equation}
        \MassFrac{solute}{C} = \frac{MW_{\text{anhydrous salt}}}{MW_{\text{hydrated salt}}}
    \end{equation}
    \item \textbf{Water Fraction in Crystal ($\MassFrac{water}{C}$):}
    \begin{equation}
        \MassFrac{water}{C} = \frac{n \times MW_{H_2O}}{MW_{\text{hydrated salt}}}
    \end{equation}
\end{itemize}
Material balances usually require solving two simultaneous equations (one for the solute, one for the solvent) to find the unknowns $\SolMass$ and $\CrysMass$.

\section{Heat Effects in Crystallization}
Thermal management is critical as crystallization involves phase changes that absorb or release heat.

\subsection{Heat of Solution ($\HeatSol$)}
Defined as the heat effect when a substance is dissolved.
\begin{itemize}
    \item \textbf{Endothermic (+):} Heat is absorbed; the system cools down. Solubility typically increases with temperature.
    \item \textbf{Exothermic (-):} Heat is released; the system heats up. Solubility typically decreases with temperature.
\end{itemize}
When converting heat of solution from molar units (kJ/kg-mol) to specific units (kJ/kg), use the molecular weight of the **crystallizing species** (e.g., the hydrated salt).

\subsection{Heat of Crystallization ($\HeatCrys$)}
This is the reverse of the heat of solution.
\begin{equation}
    \HeatCrys = - \HeatSol
\end{equation}
Since crystallization is the formation of a solid from solution, it is generally an exothermic process (releases heat) for most common salts.

\subsection{Enthalpy Balance}
The total heat transferred ($\HeatTrans$) to maintain the process temperature is calculated by an enthalpy balance:
\begin{equation}
    \HeatTrans = (\Enthalpy_2 + \Enthalpy_V) - \Enthalpy_1
\end{equation}
Where:
\begin{itemize}
    \item $\Enthalpy_1$: Enthalpy of the feed solution. The enthalpy of the feed ($H_1$) is dominated by sensible heat: $\Enthalpy_1 = \FeedMass \cdot C_p \cdot (\Temp_{feed} - \Temp_{ref})$.
    \item $\Enthalpy_2$: Enthalpy of the final mixture (crystals + mother liquor).
    \item $\Enthalpy_V$: Enthalpy of the vapor (if evaporation occurs).
    \item $\HeatTrans$: Total heat load. A positive value indicates heat must be added, while a negative value indicates heat must be removed (cooling).
\end{itemize}