% Custom Commands based on Ion Exchange Context
\newcommand{\Resin}{R}                    % Resin Matrix
\newcommand{\Capacity}{Q}                 % Total Resin Capacity
\newcommand{\ConcSol}{C}                  % Total Solution Concentration
\newcommand{\ConcResin}{q}                % Concentration in Resin Phase
\newcommand{\SelectivityCoeff}{K}         % Selectivity Coefficient
\newcommand{\Valence}{n}                  % Ion Valence
\newcommand{\Equiv}{eq}                   % Equivalents

\section{Introduction to Ion Exchange}
Ion exchange is a reversible stoichiometric chemical reaction where ions from an aqueous solution are exchanged for ions bound to a solid particle (resin)

\subsection{Applications}
Common industrial applications include:
\begin{itemize}
    \item Water softening (removal of $Ca^{2+}$ and $Mg^{2+}$)
    \item Recovery of antibiotics and vitamins from fermentation
    \item Production of high-purity water for semiconductors
    \item Wastewater treatment and heavy metal removal
\end{itemize}

\section{Ion Exchange Resins}
Resins are the solid media used for exchange. They consist of a polymer lattice (often cross-linked with divinyl benzene or DVB) with charged functional groups

\subsection{Classification of Resins}
Resins are categorized by their functional groups and the type of ions they exchange

\begin{table}[h]
\centering
\small % Added small font size
\begin{tabularx}{\columnwidth}{|l|X|X|}
\hline
\textbf{Type} & \textbf{Functional Group Example} & \textbf{Characteristics} \\ \hline
Strong-Acid Cation (SAC) & Sulfonic acid ($R-SO_3H$) & Highly ionized; behaves like strong acid. \\ \hline
Weak-Acid Cation (WAC) & Carboxylic acid ($-COOH$) & Dissociates weakly; high capacity. \\ \hline
Strong-Base Anion (SBA) & Quaternary ammonium ($-OH$) & Highly ionized; behaves like strong base. \\ \hline
Weak-Base Anion (WBA) & Amine group ($-NH$) & Efficient regeneration. \\ \hline
Chelating Resins & EDTA ($R-EDTA-Na$) & High selectivity for heavy metals. \\ \hline
\end{tabularx} % Changed to tabularx
\caption{Properties of Synthetic Resins}
\end{table}

\section{Operational Principles}

\subsection{Exchange Process}
The exchange occurs in either batch or continuous (column) modes. In water softening, Hard Water containing Calcium ($Ca^{2+}$) and Magnesium ($Mg^{2+}$) passes over a resin saturated with Sodium ($Na^+$). The hardness ions displace the sodium ions

The reaction equation for softening:
\begin{equation}
    Ca^{2+} + Na_2\Resin \leftrightarrow Ca\Resin + 2Na^+
\end{equation}

\subsection{Regeneration}
Since the reaction is reversible, the resin can be regenerated by applying a high concentration of the displaced ion to shift equilibrium back to the left
\begin{itemize}
    \item \textbf{SAC Regeneration:} Uses strong acids ($HCl$, $H_2SO_4$) or $NaCl$
    \item \textbf{SBA Regeneration:} Uses Caustic Soda ($NaOH$) or $KOH$
\end{itemize}

\section{Equilibrium Relations}
Ion exchange equilibrium is described using the Law of Mass Action.

\subsection{The Selectivity Coefficient ($\SelectivityCoeff$)}
For a generalized exchange reaction where ion $A$ in solution replaces ion $B$ on the resin:
\begin{equation}
    A^{+n} + n B^+\Resin^- \leftrightarrow A^{+n}\Resin_n^- + n B^+
\end{equation}
The selectivity coefficient $K_{A,B}$ (selectivity for A over B) is defined as
\begin{equation}
    \SelectivityCoeff_{A,B} = \frac{[\ConcResin_{AR}][\ConcSol_B]^n}{[\ConcSol_A][\ConcResin_{BR}]^n}
\end{equation}
Where:
\begin{itemize}
    \item $\ConcResin$ = Concentration in resin (equivalents/L of wet bed volume).
    \item $\ConcSol$ = Concentration in solution (equivalents/L of solution).
\end{itemize}

\subsection{Relative Selectivity}
The selectivity coefficient can also be approximated using relative molar selectivity values derived from experimental data (typically relative to $Li^+$ or $Cl^-$)
\begin{equation}
    \SelectivityCoeff_{A,B} = \frac{\SelectivityCoeff_A}{\SelectivityCoeff_B}
\end{equation}
\textbf{Affinity Series (SAC):} The resin generally prefers ions with higher valence and smaller hydrated radius
$$ Li^+ < H^+ < Na^+ < NH_4^+ < K^+ < Mg^{2+} < Ca^{2+} $$

\subsection{Conservation Equations}
To solve equilibrium problems, two conservation laws are applied 
\begin{itemize}
    \item \textbf{Total Resin Capacity ($\Capacity$):} The total number of available sites is fixed.
    \begin{equation}
        \Capacity = n \ConcResin_{AR} + \ConcResin_{BR}
    \end{equation}
    \item \textbf{Total Solution Concentration ($\ConcSol$):} The total ionic concentration in the liquid.
    \begin{equation}
        \ConcSol = n \ConcSol_A + \ConcSol_B
    \end{equation}
\end{itemize}

\section{Kinetic Model (Rate of Exchange)}
The rate of ion exchange is controlled by mass transfer resistances. The basic model involves four sequential steps 
\begin{enumerate}
    \item \textbf{Film Diffusion:} Mass transfer of ions from the bulk solution to the particle surface.
    \item \textbf{Pore Diffusion:} Diffusion of ions through the pores of the solid resin.
    \item \textbf{Surface Reaction:} Exchange of ions at the active sites on the resin lattice.
    \item \textbf{Back Diffusion:} Diffusion of the exchanged ions back out to the bulk solution.
\end{enumerate}