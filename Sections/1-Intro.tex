
\section{Core Concepts in Chemical Engineering}

\subsection{Chemical Engineering vs. Chemistry}
Chemical Engineering is a group of industrial processes in which raw materials are changed or separated into useful products.
\begin{itemize}
    \item \textbf{Chemistry:} Focuses on creating new substances, studying their properties, and investigating all possible pathways from one substance to another.
    \item \textbf{Chemical Engineering:} Focuses on designing the most optimal technology for production of a specified substance from raw materials, and developing/discovering new technological applications for materials.
\end{itemize}

\subsection{Unit Operations}
\begin{itemize}
    \item \textbf{Definition:} A unit operation is a basic step in a chemical engineering process. It's a method of analyzing and designing complex processes by breaking them down into individual, fundamental tasks.
    \item \textbf{Purpose:} To organize chemical engineering knowledge into logical groups based on the underlying physical principles.
\end{itemize}

\subsection{Classifications of Unit Operations}
Unit operations are categorized based on the primary physical process involved:
\begin{itemize}
    \item \textbf{Fluid Flow Processes:} Fluid transportation, solid fluidization.
    \item \textbf{Heat Transfer Processes:} Evaporation, condensation, heat exchange.
    \item \textbf{Mass Transfer Processes:} Absorption, adsorption, distillation, drying.
    \item \textbf{Thermodynamic Processes:} Gas liquefaction, refrigeration.
    \item \textbf{Mechanical Processes:} Solid transportation, sieving, crushing.
\end{itemize}

\section{Separation Processes}

\subsection{Purpose of Separation}
Separation is a physical process to separate a mixture into individual components. It is a critical and often costly part of chemical manufacturing, accounting for 50-90\% of total plant costs.
\begin{itemize}
    \item \textbf{Key Goals:}
    \begin{itemize}
        \item \textbf{Product Specification:} To meet the required purity of the final product.
        \item \textbf{Recovery:} To recover valuable components from a mixture.
        \item \textbf{Purification:} To remove impurities from raw materials or products.
    \end{itemize}
\end{itemize}

\subsection{Separating Homogeneous vs. Heterogeneous Mixtures}
\begin{itemize}
    \item \textbf{Heterogeneous Mixtures:} Components are not uniform and different layers or phases are often visible. Separation is typically achieved with mechanical operations.
    \begin{itemize}
        \item \textit{Examples:} Settling by gravity, flotation, centrifugal separations, filtration.
    \end{itemize}
    \item \textbf{Homogeneous Mixtures:} Components are completely uniform in a single phase. Separation requires the introduction of a \textbf{separating agent} to create a new phase.
    \begin{itemize}
        \item \textbf{Energy as an Agent:} Heat transfer creates a new phase (e.g., creating a vapor phase from a liquid in distillation).
        \item \textbf{Mass as an Agent:} A new material (a mass separating agent, MSA) is added to create a new phase (e.g., adding a solvent in liquid-liquid extraction).
    \end{itemize}
\end{itemize}

\subsection{General Separation Techniques}
Separation processes can be classified based on the technique used to induce separation:
\begin{itemize}
    \item \textbf{Separation by Phase Creation:} A new phase is created from the mixture itself, typically by Energy Separating Agents (ESA) (e.g., Distillation, Crystallization, Evaporation).
    \item \textbf{Separation by Phase Addition:} A new phase is added as a Mass Separating Agent (MSA) (e.g., Absorption, Liquid-Liquid Extraction).
    \item \textbf{Separation by Barrier Creation:} A barrier restricts the movement of certain species between phases (e.g., Membrane separation).
    \item \textbf{Separation by Solid Agent:} A solid material is used to remove solutes (e.g., Adsorption, Ion Exchange).
    \item \textbf{Separation by Force Field or Gradient:} Separation is driven by an external field (e.g., Centrifugation, Electrolysis).
\end{itemize}

\section{Classifying Separation Processes by Mechanism}

\subsection{Equilibrium Governed Processes}
These processes are designed and analyzed based on the \textbf{equilibrium stage concept}.
\begin{itemize}
    \item \textbf{Equilibrium Stage Concept:} An ideal "stage" is a device where two different phases are brought into intimate contact, allowing mass transfer to occur until the streams leaving the stage are in perfect equilibrium with each other.
    \item \textbf{Definition of Equilibrium:} A system is at equilibrium when it is stable and not changing with time under constant temperature, pressure, and composition. At equilibrium, there is \textbf{no net transfer} of components between the phases.
    \begin{itemize}
        \item \textbf{Thermal Equilibrium:} No net heat transfer (Temperatures are equal: $T^V = T^L$).
        \item \textbf{Mechanical Equilibrium:} No change in volume (Pressures are equal: $P^V = P^L$).
        \item \textbf{Chemical Equilibrium:} No net mass transfer (Chemical potentials are equal: $\mu_i^V = \mu_i^L$).
    \end{itemize}
    \item \textbf{Examples:} Distillation, absorption, 
    desorption (stripping), \\ liquid-liquid extraction, leaching, adsorption.
\end{itemize}

\subsection{Rate Governed Processes}
These processes depend on the rate of transport of a component, which is driven by a concentration gradient. The concept of equilibrium is not assumed.
\begin{itemize}
    \item \textbf{Mechanism:} Involves the migration of a substance from a region of higher concentration to lower concentration. The process is characterized by the general transport equation:
    \begin{equation}
        \text{Rate of transfer process} = \frac{\text{Driving Force}}{\text{Resistance}}
    \end{equation}
    \item \textbf{Examples:} Membrane separation, crystallization, electrolysis.
\end{itemize}

\section{Overview of Specific Mass Transfer Operations}

\begin{itemize}
    \item \textbf{Distillation:} Separation of components in a liquid mixture by boiling, based on differences in their \textbf{vapor pressure} (volatility). A vapor phase is created from a liquid phase.
    \item \textbf{Liquid-Liquid Extraction:} A solute is removed from a liquid solution by contacting it with another, immiscible liquid solvent in which the solute has a higher affinity.
    \item \textbf{Leaching (Solid-Liquid Extraction):} A solute is removed from a finely divided solid by dissolving it in a liquid solvent.
    \item \textbf{Absorption:} A solute (gas) is removed from a gas stream by dissolving it into a liquid solvent.
    \item \textbf{Adsorption:} A component from a liquid or gas stream (adsorbate) accumulates on the surface or within the pores of a solid \textbf{adsorbent}.
    \item \textbf{Membrane Separation:} A fluid mixture is separated by passing it through a semipermeable barrier (membrane) that controls the rate of movement of molecules between two phases.
    \item \textbf{Crystallization:} A solute is removed from a solution by adjusting conditions (e.g., temperature, concentration) to exceed its solubility limit, causing it to form a solid phase.
\end{itemize}

\section{Vapour-Liquid Equilibrium (VLE) Terms}
\begin{itemize}
    \item \textbf{Bubble Point Temperature:} The temperature at which the \textbf{first vapor bubble} forms when a liquid is heated slowly at constant pressure.
    \item \textbf{Dew Point Temperature:} The temperature at which the \textbf{first liquid droplet} forms when a vapor is cooled slowly at constant pressure.
    \item \textbf{Bubble Point Pressure:} The pressure at which the \textbf{first vapor bubble} forms when the pressure on a liquid is slowly reduced at constant temperature.
    \item \textbf{Dew Point Pressure:} The pressure at which the \textbf{first liquid droplet} forms when a vapor is slowly compressed at constant temperature.
\end{itemize}

\section{Design Goals for Separation Equipment}
\begin{enumerate}
    \item \textbf{Obtain the Required Products:} The primary goal is to successfully separate the feed into products that meet the required specifications (e.g., purity).
    \item \textbf{Minimize Equipment Cost:} Design equipment that works properly but is not oversized or undersized, thereby minimizing capital construction costs.
    \item \textbf{Minimize Operating Costs:} Design energy-efficient processes. Separations like distillation consume enormous amounts of energy, often up to 50\% of a plant's total operating costs.
\end{enumerate}